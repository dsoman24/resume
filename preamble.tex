% Description: This file contains the preamble for the resume. It contains the packages and commands that are used in the resume.

%  Packages
\usepackage{xparse}
\usepackage[margin=1.3cm]{geometry}
\usepackage[hidelinks]{hyperref}
\usepackage{environ}
\usepackage{newtxtext,newtxmath}
\usepackage{anyfontsize}
\usepackage{enumitem}

% Space after a section
\newcommand{\sectionSpace}{\vspace{4pt}}
% Space after an experience
\newcommand{\experienceSpace}{\vspace{4pt}}

% This command defines a custom header for a resume.
% Parameters:
%   1. Name
%   2. Email Address
%   3. Phone number
%   4. Location
%   5. Github username (i.e. the part after github.com/)
%   6. Linkedin URL name (i.e. the part after linkedin.com/in/)
\newcommand{\header}[6]{
    \begin{center}
        \textbf{\LARGE #1} \\
        #2 $|$
        #3 $|$
        #4 $|$
        \href{https://github.com/#5}{github.com/#5} $|$
        \href{https://www.linkedin.com/in/#6}{linkedin.com/in/#6}
    \end{center}
}
\newcommand{\horizontal}{\vspace{2pt}\hrule}
\newcommand{\school}[3]{\vspace{2pt}\textsc{\textbf{#1}} \hfill \textbf{\textit{#2}} \\ #3 \sectionSpace}
\newcommand{\sectitle}[1]{\vspace{3pt} \textbf{\large #1} \horizontal \vspace{4pt}}
\newcommand{\skill}[2]{\textbf{#1:} #2}

\NewDocumentEnvironment{experience}{mmmm}
    {
        \textsc{\textbf{#1}} \hfill #2

        \textit{\textbf{#3 \hfill #4}}

        \begin{itemize}[noitemsep,topsep=0pt]
    }
    {
        \end{itemize}
        \experienceSpace
    }

\NewDocumentEnvironment{subexperience}{mm}
    {

        \textit{\textbf{#1 \hfill #2}}

        \begin{itemize}[noitemsep,topsep=0pt]
    }
    {
        \end{itemize}
    }

\NewDocumentEnvironment{project}{mmmm}
    {

        \textbf{\href{#2}{#1}} $|$ \textit{#3} \hfill \textit{\textbf{#4}}

        \begin{itemize}[noitemsep,topsep=0pt]
    }
    {
        \end{itemize}
        \experienceSpace
    }

\NewDocumentEnvironment{experiencenolist}{mmmm}
    {
        \vspace{2pt}
        \textsc{\textbf{#1}} \hfill #2

        \textit{\textbf{#3 \hfill #4}}

    }{}
