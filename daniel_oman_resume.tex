\documentclass{article}

% Description: This file contains the preamble for the resume. It contains the packages and commands that are used in the resume.

%  Packages
\usepackage{xparse}
\usepackage[margin=1.3cm]{geometry}
\usepackage[hidelinks]{hyperref}
\usepackage{environ}
\usepackage{newtxtext,newtxmath}
\usepackage{anyfontsize}
\usepackage{enumitem}

\newcommand{\horizontal}{\vspace{2pt}\hrule}
\newcommand{\school}[3]{\vspace{2pt}\textsc{\textbf{#1}} \hfill \textbf{\textit{#2}} \\ #3}
\newcommand{\sectitle}[1]{\vspace{3pt} \textbf{\large #1} \horizontal}
\newcommand{\skill}[2]{\textbf{#1:} #2}

\NewDocumentEnvironment{experience}{mmmm}
    {
        \vspace{2pt}
        \textsc{\textbf{#1}} \hfill #2

        \textit{\textbf{#3 \hfill #4}}

        \begin{itemize}[noitemsep,topsep=0pt]
    }
    {
        \end{itemize}
    }

\NewDocumentEnvironment{subexperience}{mm}
    {

        \textit{\textbf{#1 \hfill #2}}

        \begin{itemize}[noitemsep,topsep=0pt]
    }
    {
        \end{itemize}
    }

\NewDocumentEnvironment{project}{mmmm}
    {

        \textbf{\href{#2}{#1}} $|$ \textit{#3} \hfill \textit{\textbf{#4}}

        \begin{itemize}[noitemsep,topsep=0pt]
    }
    {
        \end{itemize}
    }

\NewDocumentEnvironment{experiencenolist}{mmmm}
    {
        \vspace{2pt}
        \textsc{\textbf{#1}} \hfill #2

        \textit{\textbf{#3 \hfill #4}}

    }{}

\begin{document}

\thispagestyle{empty}

\header{Daniel Öman}{dsoman24@gmail.com}{(470) 553-5299}{Atlanta, GA}{dsoman24}{daniel-s-oman}

\begin{flushleft}

\sectitle{Education}

\school{Georgia Institute of Technology}{August 2021 -- May 2025 (expected)}
{\textbf{\textit{B.S. Computer Science, concentrations in Intelligence (AI/ML) and Theory}} \hfill 3.96/4.0 GPA \\ \textit{Relevant Courses}:
    Data Structures \& Algorithms,
    Machine Learning,
    Computer Organization \& Programming,
    % Objects \& Design
    Probability \& Statistics
    %Automata \& Complexity,
    %Number Theory%,
    %Deep Learning*,
    %Graduate Algorithms*
    %\textit{(* = current)}
    \\
    \textit{Activities}: Delta Chi Fraternity (Secretary, Professional Development Chair), Society of Hispanic Professional Engineers
}

\sectitle{Experience}

    \begin{experience}{Google}{Kirkland, WA}{Software Engineering Intern}{May 2024 -- August 2024}
        \item Designed and implemented a custom distributed load testing framework using C++ and Python to benchmark the scalability of the streaming metadata change-log service within Google BigQuery's core storage infrastructure.
        \item Built continuous test runs as a development workflow, leading to 70\% reduction in regressions before reaching production.
        \item Developed load sampling architecture to simulate production traffic by sending 10k+ requests per second to the read and write RPC endpoints on a variable number of BigQuery tables, exposing production bottlenecks.
        \item Designed a multi-threaded metric sampling system with C++ to compute, aggregate, and analyze latency, throughput, and error rate metrics across 25+ load sampler instances concurrently over multiple machines, yielding increased benchmark accuracy.
        \item Led efforts to fix a critical service-level objective bug affecting BigQuery's storage metadata server by implementing request retry logic, eliminating the number of error spikes by $\sim$90\%.
    \end{experience}

    \begin{experience}{Georgia Tech Efficient and Intelligent Computing Lab}{Atlanta, GA}{Undergraduate Research Assistant}{January -- May 2024}
        \item Contributed to a PyTorch toolkit used by 5+ Georgia Tech labs to train distributed Graph Neural Networks (GNNs) for applications with multiple large disjoint graphs, such as electronic design analysis and molecular modeling.
        \item Built a user-friendly modular data loading and transfer API and implemented the GraphSAGE GNN forward propagation and graph vertex embedding algorithm, improving model accuracies by an average of 15\%.
    \end{experience}

    \begin{experience}{Georgia Tech College of Computing}{Atlanta, GA}{Undergraduate Teaching Assistant (Homework Lead)}{August 2022 -- May 2024}
        \item Managed a team of 40 TAs in the development and grading of 12 homework assignments for over 800 students per semester as TA Homework Lead for CS 1331: Intro to Object-Oriented Programming (Java) under Prof.~Richard Landry and Dr.~Aibek Musaev.
        \item Led weekly recitations for 50 students; held 1-1 office hours 6 times a week to aid students with problem-solving and debugging.
    \end{experience}

    \begin{experience}{Google}{Sunnyvale, CA}{STEP Intern}{May -- August 2023}
        \item Designed, implemented, and tested an efficient parallel-processing data pipeline being used in production to provide features to train machine learning models that predict Google Workspace account upgrade, downgrade, and churn behaviors.
        \item Built pipeline using FlumeJava, a Java MapReduce framework, to extract and aggregate 70+ web domain level ML features from a database containing the HTML of more than 500 billion web pages, increasing customer coverage in the feature store by 20\%.
        \item Engineered a scalable and extensible data aggregation architecture by applying advanced OOP design patterns that reduced feature implementation time by over 50\% and provided an intuitive interface for future feature store contributions.
        \item Refactored pipeline to improve reliability by developing a system to flush intermediate output to a Spanner database across 10k+ processes during a full table scan, preventing up to 7 days worth of lost data for each pipeline run.
    \end{experience}

    \sectitle{Projects}

    \vspace{3pt}

    \begin{project}{Hemodynamics Calculator}{https://github.com/dsoman24/hemodynamics-calculator}{JavaScript, ReactJS, MongoDB, Express, NodeJS}{August 2023 -- April 2024}
        \item Developed the Hemodynamics Calculator, a full-stack MERN application for the Emory University School of Medicine used by over 10 clinicians to reduce blood flow measurement error daily, critically impacting more than 1,000 cardiac ICU patients a year.
        \item Placed 3rd out of 50 teams in the Georgia Tech CS Capstone Expo, presenting to 40+ industry professionals and professors.
    \end{project}

    \begin{project}{Machine Learning Soccer Prediction}{https://dsoman24.github.io/ml-project/}{Python, Scikit-Learn, PyTorch, NumPy, Matplotlib, Seaborn}{August -- December 2023}
        \item Worked on a team of 5 to build and train logistic regression, random forest, and artificial neural network models using Scikit-Learn and PyTorch to predict soccer match outcomes with 70\% accuracy, beating benchmark betting odds data by 8\%.
        \item Built feature engineering strategies and conducted hyperparameter tuning to reduce overfitting, improving accuracy by $\sim$10\%.
    \end{project}

\sectitle{Skills}

    \vspace{3pt}
    \skill{Programming Languages}{Java, C/C++, Python, SQL, JavaScript, \LaTeX} \\
    \skill{Frameworks}{FlumeJava (MapReduce), JUnit, NumPy, Pandas, Scikit-Learn, PyTorch, ReactJS, Express, NodeJS, Flask} \\
    \skill{Tools}{Git, Mercurial, Bazel, Protobuf, gRPC, Spanner, MySQL, MongoDB} \\

\end{flushleft}

\end{document}