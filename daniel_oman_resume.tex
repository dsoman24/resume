\documentclass{article}
\usepackage{xparse}

\usepackage[margin=1.3cm]{geometry}
\usepackage[hidelinks]{hyperref}
\usepackage{environ}
\usepackage{newtxtext,newtxmath}
\usepackage{anyfontsize}
\usepackage{enumitem}

\newcommand{\horizontal}{\vspace{2pt}\hrule}
\newcommand{\school}[3]{\vspace{2pt}\textsc{\textbf{#1}} \hfill \textbf{\textit{#2}} \\ #3}
\newcommand{\sectitle}[1]{\vspace{3pt} \textbf{\large #1} \horizontal}
\newcommand{\skill}[2]{\textbf{#1:} #2}

% MACROS

\NewDocumentEnvironment{experience}{mmmm}
    {
        \vspace{2pt}
        \textsc{\textbf{#1}} \hfill #2

        \textit{\textbf{#3 \hfill #4}}

        \begin{itemize}[noitemsep,topsep=0pt]
    }
    {
        \end{itemize}
    }

\NewDocumentEnvironment{subexperience}{mm}
    {

        \textit{\textbf{#1 \hfill #2}}

        \begin{itemize}[noitemsep,topsep=0pt]
    }
    {
        \end{itemize}
    }

\NewDocumentEnvironment{project}{mmmm}
    {

        \textbf{\href{#2}{#1}} $|$ \textit{#3} \hfill \textit{\textbf{#4}}

        \begin{itemize}[noitemsep,topsep=0pt]
    }
    {
        \end{itemize}
    }

\NewDocumentEnvironment{experience_no_list}{mmmm}
    {
        \vspace{2pt}
        \textsc{\textbf{#1}} \hfill #2

        \textit{\textbf{#3 \hfill #4}}

    }{}

\begin{document}
\thispagestyle{empty}
\begin{center}
    \textbf{\LARGE Daniel Öman} \\
    dsoman24@gmail.com $|$ (470) 553-5299 $|$ Atlanta, GA \\
    \href{https://github.com/dsoman24}{github.com/dsoman24} $|$ \href{https://www.linkedin.com/in/daniel-s-oman/}{www.linkedin.com/in/daniel-s-oman}
\end{center}

\begin{flushleft}
\sectitle{Education}

\school{Georgia Institute of Technology}{August 2021 -- May 2025 (expected)}
{\textbf{\textit{B.S. Computer Science, concentrations in Intelligence (AI/ML) and Theory}} \hfill 4.0/4.0 GPA \\ \textit{Relevant Coursework}: Data Structures \& Algorithms, Computer Organization \& Programming, Probability \& Statistics, Combinatorics, Linear Algebra \& Vector Spaces, Machine Learning*, Design \& Analysis of Algorithms* (\textit{* = current})}

\sectitle{Experience}

    \begin{experience}{Google}{Sunnyvale, CA}{STEP Intern}{May -- August 2023}
        \item Built and tested an efficient parallel-processing data pipeline using FlumeJava, a Java MapReduce framework, to extract and aggregate over 70 web domain level ML features from a database containing the HTML of more than 500 billion web pages.
        \item Designed and implemented a scalable and extensible data aggregation architecture by applying advanced OOP design patterns that reduced feature implementation time by over 50\% and provided an intuitive interface for future feature store contributions.
        \item Refactored pipeline to improve reliability by developing a system to flush intermediate output to disk across 100k+ threads during a full table scan, preventing data loss and saving more than 7 days of progress during each pipeline execution.
        \item New feature store increased customer coverage by 20\% over the existing store and is used in production to train machine learning models utilized for predicting Google Workspace account upgrade, downgrade, and churn behaviors.
    \end{experience}

    \begin{experience}{Georgia Tech Financial Services and Innovation Lab}{Atlanta, GA}{Undergraduate Researcher}{January -- May 2023}
        \item Led a team of 4 researchers in performing sentiment analysis on earnings calls transcripts on 12 electric vehicle companies using the large language model FinBERT and natural language processing library NLTK.
        \item Developed a custom web scraper using Beautiful Soup to extract over 70 earnings call transcripts from The Motley Fool.
        \item Created dynamic visualizations from analyzed text data to conclude that 5 major US government policies drove spikes in positive sentiment in earning calls from companies that focus on electric vehicle production.
    \end{experience}

    \begin{experience}{Georgia Tech College of Computing}{Atlanta, GA}{Undergraduate Teaching Assistant (Homework Lead)}{August 2022 -- Present}
        \item Manage a team of 40 TAs in the development and grading of 12 homework assignments per semester for over 800 students as TA Homework Lead for CS 1331: Intro to Object-Oriented Programming under Prof.~Richard Landry and Dr.~Aibek Musaev.
        \item Lead weekly recitations for 50 students and help students with problem-solving and debugging during one-on-one office hours.
        \item Grade 4 exams per semester and write auto-grader unit tests for assignments using the Java Reflections library.
    \end{experience}

    \begin{experience}{Ermi}{Atlanta, GA}{Engineering Intern}{July -- August 2021}
        \item Analyzed data and created decision trees from health insurance claims data from over 1000 knee surgery patients using R.
        \item Identified the highest cost patients to target for non-surgical intervention.
    \end{experience}
    \begin{subexperience}{Engineering Intern}{July -- August 2019}
        \item Analyzed 10k+ data points from a robot that diagnoses knee injuries, with analysis to be incorporated into research papers.
        \item Learned and used R to organize and visualize datasets in over 40 plots to assess the reliability and accuracy of the robot.
    \end{subexperience}

    \begin{experience}{Georgia Tech Research Institute}{Atlanta, GA}{Research Intern}{June -- July 2020}
        \item Worked in a team of 4 to develop an app that creates a Bluetooth mesh network for emergency communication.
        \item Implemented routing algorithms in Python and Java and ran simulations of the app to investigate network properties and stability.
    \end{experience}

\sectitle{Projects}

    \vspace{3pt}

    \begin{project}{Minesweeper Probabilistic Strategy}{https://github.com/dsoman24/minesweeper}{Java, JavaFX, Python, Pandas, Jupyter Notebook}{December 2022 -- July 2023}
        \item Developed a probabilistic algorithm in Java to solve Minesweeper games with 96\%, 80\%, and 30\% win rates for easy, medium, and hard difficulties, significantly higher than the approximate 46\%, 22\%, and 13\% respective human win rates.
        \item Built row reduction and tree-traversal algorithms to reduce game state matrix dimensionality, lowering solution time by over 30\%.
        % \item Implemented gameplay from scratch in Java and used JavaFX for a responsive graphical user interface.
    \end{project}

    \begin{project}{Ruter-Sju Card Game Bot and Monte Carlo Simulation}{https://github.com/dsoman24/ruter-sju}{Python, Pandas, NumPy, PyPlot}{December 2021 -- March 2022}
        \item Designed and implemented algorithms in Python to play card game Ruter-Sju to investigate best game strategy.
        \item Built a Monte Carlo simulation with 20k+ games and used Pandas and PyPlot libraries to analyze and visualize game data.
    \end{project}

\sectitle{Skills}

    \vspace{3pt}
    \skill{Technologies}{Java (Including JavaFX, Android Studio), Python (Including Pandas, NumPy, Beautiful Soup), C, Git, LaTeX, SQL, R} \\
    \skill{Languages}{Fluent in Spanish, Swedish, English} \\
    % \skill{Affiliations}{Delta Chi Fraternity (Scholarship Chair), Society of Hispanic Professional Engineers, Consult Your Community}

\end{flushleft}

\end{document}