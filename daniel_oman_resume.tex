\documentclass{article}
\usepackage{xparse}

\usepackage[margin=1.3cm]{geometry}
\usepackage[hidelinks]{hyperref}
\usepackage{environ}
\usepackage{newtxtext,newtxmath}
\usepackage{anyfontsize}
\usepackage{enumitem}

\newcommand{\horizontal}{\vspace{2pt}\hrule}
\newcommand{\school}[3]{\vspace{2pt}\textsc{\textbf{#1}} \hfill \textbf{\textit{#2}} \\ #3}
\newcommand{\sectitle}[1]{\vspace{3pt} \textbf{\large #1} \horizontal}
\newcommand{\skill}[2]{\textbf{#1:} #2}

% MACROS

\NewDocumentEnvironment{experience}{mmmm}
    {
        \vspace{2pt}
        \textsc{\textbf{#1}} \hfill #2

        \textit{\textbf{#3 \hfill #4}}

        \begin{itemize}[noitemsep,topsep=0pt]
    }
    {
        \end{itemize}
    }

\NewDocumentEnvironment{subexperience}{mm}
    {

        \textit{\textbf{#1 \hfill #2}}

        \begin{itemize}[noitemsep,topsep=0pt]
    }
    {
        \end{itemize}
    }

\NewDocumentEnvironment{project}{mmmm}
    {

        \textbf{\href{#2}{#1}} $|$ \textit{#3} \hfill \textit{\textbf{#4}}

        \begin{itemize}[noitemsep,topsep=0pt]
    }
    {
        \end{itemize}
    }

\NewDocumentEnvironment{experiencenolist}{mmmm}
    {
        \vspace{2pt}
        \textsc{\textbf{#1}} \hfill #2

        \textit{\textbf{#3 \hfill #4}}

    }{}

\begin{document}
\thispagestyle{empty}
\begin{center}
    \textbf{\LARGE Daniel Öman} \\
    dsoman24@gmail.com $|$ (470) 553-5299 $|$ Atlanta, GA \\
    \href{https://github.com/dsoman24}{github.com/dsoman24} $|$ \href{https://www.linkedin.com/in/daniel-s-oman/}{www.linkedin.com/in/daniel-s-oman}
\end{center}

\begin{flushleft}
\sectitle{Education}

\school{Georgia Institute of Technology}{August 2021 -- May 2025 (expected)}
{\textbf{\textit{B.S. Computer Science, concentrations in Intelligence (AI/ML) and Theory}} \hfill 4.0/4.0 GPA \\ \textit{Relevant Coursework}: Data Structures \& Algorithms, Machine Learning, Design \& Analysis of Algorithms, Computer Organization \& Programming, Probability \& Statistics, Combinatorics, Linear Algebra, Automata \& Complexity*, Number Theory* (\textit{* = current})}

\sectitle{Experience}

    \begin{experience}{Georgia Tech Efficient and Intelligent Computing Lab}{Atlanta, GA}{Undergraduate Research Assistant}{January 2024 -- Present}
        \item Develop a PyTorch toolkit to train distributed Graph Neural Networks (GNNs) for applications with multiple disjoint large graphs, such as electronic design automation analysis.
        \item Build a user-friendly modular data loading and transfer API and implement the GraphSAGE GNN forward propagation and graph vertex embedding algorithm to improve model accuracies by an average of 15\%.
    \end{experience}

    \begin{experience}{Google}{Sunnyvale, CA}{Software Engineering (STEP) Intern}{May -- August 2023}
        \item Implemented and tested an efficient parallel-processing data pipeline being used in production to train machine learning models that predict Google Workspace account upgrade, downgrade, and churn behaviors.
        \item Built pipeline using FlumeJava, a Java MapReduce framework, to extract and aggregate 70+ web domain level ML features from a database containing the HTML of more than 500 billion web pages, increasing customer coverage in the feature store by 20\%.
        \item Designed and implemented a scalable and extensible data aggregation architecture by applying advanced OOP design patterns that reduced feature implementation time by over 50\% and provided an intuitive interface for future feature store contributions.
        \item Refactored pipeline to improve reliability by developing a system to flush intermediate output to disk across 100k+ threads during a full table scan, preventing data loss by storing more than 7 days of data progress during each pipeline execution.
    \end{experience}

    \begin{experience}{Georgia Tech College of Computing}{Atlanta, GA}{Undergraduate Teaching Assistant (Homework Lead)}{August 2022 -- Present}
        \item Manage a team of 40 TAs in the development and grading of 12 homework assignments per semester for over 800 students as TA Homework Lead for CS 1331: Intro to Object-Oriented Programming under Prof.~Richard Landry and Dr.~Aibek Musaev.
        \item Lead weekly recitations for 50 students and help students with problem-solving and debugging during one-on-one office hours.
        \item Grade 4 exams per semester and write auto-grader unit tests for assignments using the Java Reflections library.
    \end{experience}

    \begin{experience}{Georgia Tech Financial Services and Innovation Lab}{Atlanta, GA}{Undergraduate Research Assistant (Team Lead)}{January -- May 2023}
        \item Led a team of 4 researchers in performing sentiment analysis on earnings calls transcripts on 12 electric vehicle companies using the large language model FinBERT and natural language processing library NLTK.
        \item Developed a custom web scraper using Beautiful Soup to extract over 70 earnings call transcripts from The Motley Fool.
        \item Created dynamic visualizations from analyzed text data to conclude that 5 major US government policies drove spikes in positive sentiment in earning calls from companies that focus on electric vehicle production.
    \end{experience}

\sectitle{Projects}

    \vspace{3pt}

    \begin{project}{Georgia Tech Computer Science Capstone Project}{https://github.com/NamkhangNLe/hemodynamics-calculator}{JavaScript, ReactJS, MongoDB, ExpressJS, NodeJS}{August 2023 -- Present}
        \item Develop the Hemodynamics Calculator, a full-stack application for the Emory University School of Medicine, to be used by 10 clinicians to reduce measurement error daily, impacting over 1,000 patients.
        \item Leverage ReactJS and MongoDB to develop user-friendly interactive graphics and visualizations of trends in patient data.
    \end{project}

    \begin{project}{Machine Learning Soccer Prediction}{https://dsoman24.github.io/ml-project/}{Python, sklearn, PyTorch, NumPy, Matplotlib, Seaborn}{August -- December 2023}
        \item Worked on a team of 5 to build and train Logistic Regression, Random Forest, and Artificial Neural Network models to predict soccer match outcomes with 70\% accuracy.
        \item Developed feature engineering strategies, performed dimensionality reduction, and conducted hyperparameter tuning to reduce overfitting, improving model accuracy by $\sim$10\%.
    \end{project}

    \begin{project}{Minesweeper Probabilistic Solver}{https://github.com/dsoman24/minesweeper}{Java, JavaFX, Python, Pandas, Jupyter Notebook}{December 2022 -- July 2023}
        \item Developed a probabilistic algorithm in Java to solve Minesweeper games with 96\%, 80\%, and 30\% win rates for easy, medium, and hard difficulties, significantly higher than the approximate 46\%, 22\%, and 13\% respective human win rates.
        \item Built row reduction and tree-traversal algorithms to reduce game state matrix dimensionality, lowering solution time by over 30\%.
    \end{project}

\sectitle{Skills}

    \vspace{3pt}
    \skill{Technologies}{Java (Including JavaFX, JUnit, Android Studio), Python (Pandas, NumPy, sklearn, PyTorch), C, SQL, Git, LaTeX, R} \\
    \skill{Languages}{Fluent in Spanish, Swedish, English} \\
    \skill{Affiliations}{Delta Chi Fraternity (Secretary, Professional Development Chair), Society of Hispanic Professional Engineers}

\end{flushleft}

\end{document}