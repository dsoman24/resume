\documentclass{article}

% Description: This file contains the preamble for the resume. It contains the packages and commands that are used in the resume.

%  Packages
\usepackage{xparse}
\usepackage[margin=1.3cm]{geometry}
\usepackage[hidelinks]{hyperref}
\usepackage{environ}
\usepackage{newtxtext,newtxmath}
\usepackage{anyfontsize}
\usepackage{enumitem}

\newcommand{\horizontal}{\vspace{2pt}\hrule}
\newcommand{\school}[3]{\vspace{2pt}\textsc{\textbf{#1}} \hfill \textbf{\textit{#2}} \\ #3}
\newcommand{\sectitle}[1]{\vspace{3pt} \textbf{\large #1} \horizontal}
\newcommand{\skill}[2]{\textbf{#1:} #2}

\NewDocumentEnvironment{experience}{mmmm}
    {
        \vspace{2pt}
        \textsc{\textbf{#1}} \hfill #2

        \textit{\textbf{#3 \hfill #4}}

        \begin{itemize}[noitemsep,topsep=0pt]
    }
    {
        \end{itemize}
    }

\NewDocumentEnvironment{subexperience}{mm}
    {

        \textit{\textbf{#1 \hfill #2}}

        \begin{itemize}[noitemsep,topsep=0pt]
    }
    {
        \end{itemize}
    }

\NewDocumentEnvironment{project}{mmmm}
    {

        \textbf{\href{#2}{#1}} $|$ \textit{#3} \hfill \textit{\textbf{#4}}

        \begin{itemize}[noitemsep,topsep=0pt]
    }
    {
        \end{itemize}
    }

\NewDocumentEnvironment{experiencenolist}{mmmm}
    {
        \vspace{2pt}
        \textsc{\textbf{#1}} \hfill #2

        \textit{\textbf{#3 \hfill #4}}

    }{}

\begin{document}

\thispagestyle{empty}

\header{Daniel Öman}{dsoman24@gmail.com}{(470) 553-5299}{Atlanta, GA}{dsoman24}{daniel-s-oman}

\begin{flushleft}

\sectitle{Education}

\school{Georgia Institute of Technology}{August 2021 -- May 2025 (expected)}
{\textbf{\textit{B.S. Computer Science, concentrations in Intelligence (AI/ML) and Theory}} \hfill 3.96/4.0 GPA \\ \textit{Relevant Coursework}: Data Structures \& Algorithms, Machine Learning, Deep Learning*, Graduate Algorithms*, Computer Organization \& Programming, Probability \& Statistics, Automata \& Complexity, Number Theory \textit{(* = current)}}

\sectitle{Experience}

    % \begin{experiencenolist}{\href{https://www.playerzero.ai/}{PlayerZero}}{Atlanta, GA}{Software Engineering Intern}{August -- December 2024}
    % \end{experiencenolist}

    \begin{experience}{Google}{Kirkland, WA}{Software Engineering Intern}{May -- August 2024}
        \item Design and implement a distributed load testing framework using C++ and Python to gauge scalability of a novel data streaming service and write-ahead log within Google BigQuery's core infrastructure.
        \item Implement continuous runs of the framework as a custom workflow to be used in development to detect bottlenecks and prevent latency and throughput regressions before reaching production, leading to a projected 20\% reduction in server request latency.
        \item New load sampling framework reduces test implementation time by 50\%, enabling faster iteration cycles for performance testing.
        \item Developed custom monitoring dashboards, allowing for 4x faster detection of critical performance regressions.
    \end{experience}

    \begin{experience}{Georgia Tech Efficient and Intelligent Computing Lab}{Atlanta, GA}{Undergraduate Research Assistant}{January -- May 2024}
        \item Contributed to a PyTorch toolkit to train distributed Graph Neural Networks (GNNs) for applications with multiple disjoint large graph datasets, such as electronic design automation analysis.
        \item Built a user-friendly modular data loading and transfer API and implemented the GraphSAGE GNN forward propagation and graph vertex embedding algorithm, improving model accuracies by an average of 15\%.
    \end{experience}

    \begin{experience}{Georgia Tech College of Computing}{Atlanta, GA}{Undergraduate Teaching Assistant (Homework Lead)}{August 2022 -- May 2024}
        \item Managed a team of 40 TAs in the development and grading of 12 homework assignments for over 800 students per semester as TA Homework Lead for CS 1331: Intro to Object-Oriented Programming (Java) under Prof.~Richard Landry and Dr.~Aibek Musaev.
        \item Led weekly recitations for 50 students and helped students with problem-solving and debugging during one-on-one office hours.
    \end{experience}

    \begin{experience}{Google}{Sunnyvale, CA}{Software Engineering (STEP) Intern}{May -- August 2023}
        \item Implemented and tested an efficient parallel-processing data pipeline being used in production to train machine learning models that predict Google Workspace account upgrade, downgrade, and churn behaviors.
        \item Built pipeline using FlumeJava, a Java MapReduce framework, to extract and aggregate 70+ web domain level ML features from a database containing the HTML of more than 500 billion web pages, increasing customer coverage in the feature store by 20\%.
        \item Engineered a scalable and extensible data aggregation architecture by applying advanced OOP design patterns that reduced feature implementation time by over 50\% and provided an intuitive interface for future feature store contributions.
        \item Refactored pipeline to improve reliability by developing a system to flush intermediate output to a Spanner database across 100k+ threads during a full table scan, preventing data loss by storing more than 7 days of data progress during each pipeline execution.
    \end{experience}

    \begin{experience}{Georgia Tech Financial Services and Innovation Lab}{Atlanta, GA}{Undergraduate Research Assistant (Team Lead)}{January -- May 2023}
        \item Led a team of 4 researchers developing a Python pipeline to perform sentiment analysis on earnings calls transcripts on 12 electric vehicle companies using the large language model FinBERT and natural language processing library NLTK.
        \item Developed a custom web scraper using Beautiful Soup to extract over 70 earnings call transcripts from The Motley Fool.
        % \item Created dynamic visualizations from analyzed text data to conclude that 5 major US government policies drove spikes in positive sentiment in earning calls from companies that focus on electric vehicle production.
    \end{experience}


    \sectitle{Projects}

    \vspace{3pt}

    \begin{project}{Hemodynamics Calculator}{https://github.com/dsoman24/hemodynamics-calculator}{JavaScript, ReactJS, MongoDB, Express, NodeJS}{August 2023 -- April 2024}
        \item Developed the Hemodynamics Calculator, a full-stack MERN application for the Emory University School of Medicine used by over 10 clinicians to reduce blood flow measurement error daily, critically impacting more than 1,000 cardiac ICU patients a year.
        \item Placed 3rd out of 50 teams in the Georgia Tech CS Capstone Expo, presenting to 40+ industry professionals and professors.
    \end{project}

    \begin{project}{Machine Learning Soccer Prediction}{https://dsoman24.github.io/ml-project/}{Python, sklearn, PyTorch, NumPy, Matplotlib, Seaborn}{August -- December 2023}
        \item Worked on a team of 5 to build and train logistic regression, random forest, and artificial neural network models using Scikit-Learn and PyTorch to predict soccer match outcomes with 70\% accuracy, beating benchmark betting odds data by 8\%.
        \item Built feature engineering strategies and conducted hyperparameter tuning to reduce overfitting, improving accuracy by $\sim$10\%.
    \end{project}

\sectitle{Skills}

    \vspace{3pt}
    \skill{Programming Languages}{Java, C/C++, Python, SQL, JavaScript, LaTeX} \\
    \skill{Frameworks}{FlumeJava (MapReduce), JUnit, NumPy, Pandas, Scikit-Learn, PyTorch, ReactJS, NodeJS, Flask} \\
    \skill{Tools}{Git, Mercurial, Bazel, Protobuf, gRPC, Spanner, MySQL, MongoDB} \\
    % \skill{Languages}{Fluent in Spanish, Swedish, and English} \\

\end{flushleft}

\end{document}